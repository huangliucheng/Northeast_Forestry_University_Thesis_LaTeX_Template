% !Mode:: “TeX:UTF-8”
% !TeX root = ./main.tex
%%%%%%%%%%%%%%%%%%%%%%%%%%%%%%%%%%%%%%%%%%%%%%%%%%%%%%%%%%%%%%%%%%%%%%%%%%%%%%%%%%%%%%%%
%%%%%%%%%%%%%%%%%%%%%%%%%%%东北林业大学%%理学院%%本科生毕业论文LaTeX模板%%%%%%%%%%%%%%%%%%%%%
%%%%%%%%%%%%%%%%%%%%%%%%%%%%%%%%%%%%%%%%%%%%%%%%%%%%%%%%%%%%%%%%%%%%%%%%%%%%%%%%%%%%%%%%
% Copyright 2022-2025 LiYu
%
% This work may be distributed and/or modified under the
% conditions of the LaTeX Project Public License, either version 1.3
% of this license or (at your option) any later version.
%
%
% The Current Maintainer of this work is Li Yu of collage of science in NEFU.
% If you have any questions about the template, please contact------liy@nefu.edu.cn----
%%%%%%%%%%%%%%%%%%%%%%%%%%%%%%%%%%%%%%%%%%%%%%%%%%%%%%%%%%%%%%%%%%%%%%%%%%%%%%%%%%%%%%%%
%注意编译方法:
%
% Compile with: xelatex -> biber -> xelatex -> xelatex
%
%此模板供东北林业大学学生免费试用,版权归理学院李雨老师所有,任何人修改发布请标明以上说明。
%=======================================================================================%
%此模板基于理学院李雨老师的毕业论文Latex模板进行修改, 对原模板进行了封装和简化
%=======================================================================================%

%==================================================%
% 导言区, 如果有额外需要使用的包, 请在此处引用
%==================================================%
%\usepackage{}


%==================================================%
% 论文相关信息
%==================================================%


\documentclass{nefuthesis}
\title{论文标题}
\author{论文作者}
\supervisor{指导老师}{指导老师职称}
\degreeinfo{本科毕业论文} % 本科毕业论文 or 硕士毕业论文
\college{学院}
\major{专业}{班级}
\studentid{学号}
\date{2024年6月}
\begin{document}
% 显示封面
\makecover

% 中文摘要环境, 请在此环境内编写中文摘要
\begin{abstract}[cn][1]{关键词1;关键词2;关键词3}
这是中文摘要的内容,摘要的内容一般包括研究背景、研究目的、研究方法、研究结果和结论等。摘要应简明扼要,突出重点,
避免使用过于专业的术语和缩略语。中文摘要一般在300字左右。
\end{abstract}

% 英文摘要环境, 请在此环境内编写英文摘要
\begin{abstract}[en][2]{Keyword1;\sep Keyword2;\sep Keyword3}
This is English abstract, the content of the abstract generally includes research background, 
research purpose, research methods, research results and conclusions. The abstract should be concise 
and highlight the key points, avoiding overly professional terms and abbreviations. The English abstract 
is generally about 300 words.
\end{abstract}

% 目录
\content

% 论文正文环境
\begin{thesis}
    % 建议按照以下形式进行正文内容的编写
    % 将所有的章节放在sections文件夹中, 每一章节使用的图片按照pictures->chapter number->image name的形式进行命名
    % 例如:第一章的图片放在pictures->chapter1->image name.png中
    % \section{一级标题}
    % \input{sections/chapter2.tex}
    % \input{sections/chapter3.tex}

    \section{一级标题1}
    \subsection{二级标题1}
    \subsubsection{三级标题1}

    \section{一级标题2}
    \subsection{二级标题2}
    \subsubsection{三级标题2}

\end{thesis} 

% 附录环境, 请在此环境内编写附录
\begin{Appendix}
    % \input{sections/appendix.tex}
\end{Appendix}

% 结论环境, 请在此环境内编写结论
\begin{conclusion}
    % \input{sections/conclusion.tex}
\end{conclusion}

\references{references.bib} % 参考文献

% \begin{biblist}
%     \bibitem{TSHU}张三. 图书名[M]. 第一版. 哈尔滨: 东北林业大学出版社, 2023: 1-10.
%     \bibitem{QKLW}张三. 期刊论文名[J]. 东北林业大学自然科学学报. 2023, 1(2):1-3.
%     \bibitem{HYLW}张三. 会议论文名[C]. 李四(编者名). 东北林业大学数学会议, 2023. 哈尔滨, 东北林业大学出版社, 2023: 1-3.
%     \bibitem{XWLW}张三. 硕士论文名[D]. 东北林业大学硕士论文. 2023:1-3. 
%     \bibitem{ZZWS2020}Zhu N, Zhang D, Wang W, et al. A novel coronavirus from patients with pneumonia in China[J]. New England Journal of Medicine. 2020, 382: 727-733.
%     \bibitem{ZDX2021}Zhang H, Du F, Cao X, et al. Clinical characteristics of coronavirus disease 2019 (COVID-19) in patients out of Wuhan from China: a case control study[J]. BMC Infectious Diseases. 2021, 21: 207.
%     \bibitem{HBJ2022}Hassan M A, Bala A A, Jatau A I. Low rate of COVID-19 vaccination in Africa: a cause for concern[J]. Therapeutic Advances in Vaccines Immunotherapy. 2022, 10: 1-3.
% \end{biblist}

% 致谢环境, 请在此环境内编写致谢
\begin{acknowledgement}
    % \input{sections/acknowledgement.tex}
\end{acknowledgement}

% % 原创性声明和版权授权书命令
\declaration

\end{document} 